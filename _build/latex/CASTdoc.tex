%% Generated by Sphinx.
\def\sphinxdocclass{report}
\documentclass[letterpaper,10pt,english]{sphinxmanual}
\ifdefined\pdfpxdimen
   \let\sphinxpxdimen\pdfpxdimen\else\newdimen\sphinxpxdimen
\fi \sphinxpxdimen=.75bp\relax

\PassOptionsToPackage{warn}{textcomp}
\usepackage[utf8]{inputenc}
\ifdefined\DeclareUnicodeCharacter
% support both utf8 and utf8x syntaxes
  \ifdefined\DeclareUnicodeCharacterAsOptional
    \def\sphinxDUC#1{\DeclareUnicodeCharacter{"#1}}
  \else
    \let\sphinxDUC\DeclareUnicodeCharacter
  \fi
  \sphinxDUC{00A0}{\nobreakspace}
  \sphinxDUC{2500}{\sphinxunichar{2500}}
  \sphinxDUC{2502}{\sphinxunichar{2502}}
  \sphinxDUC{2514}{\sphinxunichar{2514}}
  \sphinxDUC{251C}{\sphinxunichar{251C}}
  \sphinxDUC{2572}{\textbackslash}
\fi
\usepackage{cmap}
\usepackage[T1]{fontenc}
\usepackage{amsmath,amssymb,amstext}
\usepackage{babel}



\usepackage{times}
\expandafter\ifx\csname T@LGR\endcsname\relax
\else
% LGR was declared as font encoding
  \substitutefont{LGR}{\rmdefault}{cmr}
  \substitutefont{LGR}{\sfdefault}{cmss}
  \substitutefont{LGR}{\ttdefault}{cmtt}
\fi
\expandafter\ifx\csname T@X2\endcsname\relax
  \expandafter\ifx\csname T@T2A\endcsname\relax
  \else
  % T2A was declared as font encoding
    \substitutefont{T2A}{\rmdefault}{cmr}
    \substitutefont{T2A}{\sfdefault}{cmss}
    \substitutefont{T2A}{\ttdefault}{cmtt}
  \fi
\else
% X2 was declared as font encoding
  \substitutefont{X2}{\rmdefault}{cmr}
  \substitutefont{X2}{\sfdefault}{cmss}
  \substitutefont{X2}{\ttdefault}{cmtt}
\fi


\usepackage[Bjarne]{fncychap}
\usepackage[,numfigreset=1,mathnumfig]{sphinx}

\fvset{fontsize=\small}
\usepackage{geometry}


% Include hyperref last.
\usepackage{hyperref}
% Fix anchor placement for figures with captions.
\usepackage{hypcap}% it must be loaded after hyperref.
% Set up styles of URL: it should be placed after hyperref.
\urlstyle{same}

\addto\captionsenglish{\renewcommand{\contentsname}{Quick online usage and offline Installation}}

\usepackage{sphinxmessages}




\title{CAST - A site assessment tool}
\date{Apr 12, 2021}
\release{}
\author{P.\@{} K.\@{} Yadav, S.\@{} Birla, V.\@{} Baliga, A.\@{} Köhler, K.\@{} Aryal and others}
\newcommand{\sphinxlogo}{\vbox{}}
\renewcommand{\releasename}{}
\makeindex
\begin{document}

\pagestyle{empty}
\sphinxmaketitle
\pagestyle{plain}
\sphinxtableofcontents
\pagestyle{normal}
\phantomsection\label{\detokenize{intro::doc}}



\bigskip\hrule\bigskip


\sphinxAtStartPar
Contamination Assessment and Site\sphinxhyphen{}management Tool (CAST) \sphinxhyphen{} A browser based tool for site assessment \#\#

\sphinxAtStartPar
\sphinxstylestrong{Prabhas} will do this \sphinxhyphen{} actually it is the readme file in the website currently.


\chapter{Using CAST Online (still in development)}
\label{\detokenize{contents/online/online_usage:using-cast-online-still-in-development}}\label{\detokenize{contents/online/online_usage::doc}}
\sphinxAtStartPar
This is introduction to CAST online interface.

\sphinxAtStartPar
\sphinxstylestrong{Anton} and \sphinxstylestrong{Vedanti} can do this.


\section{User Login and Access}
\label{\detokenize{contents/online/login:user-login-and-access}}\label{\detokenize{contents/online/login::doc}}
\sphinxAtStartPar
\sphinxstylestrong{Vedanti} will do this.
\begin{enumerate}
\sphinxsetlistlabels{\arabic}{enumi}{enumii}{}{.}%
\item {} 
\sphinxAtStartPar
Why Login is required

\item {} 
\sphinxAtStartPar
What are accessible without login

\item {} 
\sphinxAtStartPar
What user info are stored and if they are cross\sphinxhyphen{}verified.

\item {} 
\sphinxAtStartPar
Anything more

\end{enumerate}


\section{A quick usage example of CAST}
\label{\detokenize{contents/online/quick_example:a-quick-usage-example-of-cast}}\label{\detokenize{contents/online/quick_example::doc}}
\sphinxAtStartPar
\sphinxstylestrong{Sandhya} and \sphinxstylestrong{Iram} to do this.

\sphinxAtStartPar
This is one very simple example and linking to the model page for more detailed example.


\chapter{Installation of browser\sphinxhyphen{}based CAST}
\label{\detokenize{contents/offline/offline_installation:installation-of-browser-based-cast}}\label{\detokenize{contents/offline/offline_installation::doc}}
\sphinxAtStartPar
\sphinxstylestrong{Kanishk} will do this with help from \sphinxstylestrong{Vedanti} and \sphinxstylestrong{Prabhas}

\sphinxAtStartPar
This is mostly already done. We need to reformat and that’s all.


\section{Quick example of offline CAST}
\label{\detokenize{contents/offline/quick_example:quick-example-of-offline-cast}}\label{\detokenize{contents/offline/quick_example::doc}}
\sphinxAtStartPar
\sphinxstylestrong{Sandhya} and \sphinxstylestrong{Iram} to do this.

\sphinxAtStartPar
This is one very simple example and linking to the model page for more detailed example.


\section{Updating CAST}
\label{\detokenize{contents/offline/updating:updating-cast}}\label{\detokenize{contents/offline/updating::doc}}
\sphinxAtStartPar
The following steps must be taken.

\sphinxAtStartPar
\sphinxstylestrong{Kanishk Aryal} with help from \sphinxstylestrong{Vedanti} to do this

\sphinxAtStartPar
This means how to update the CAST when the software updates. Nothing much here. E.g., update database etc.


\chapter{CAST Toolbox \sphinxhyphen{} Database Models}
\label{\detokenize{contents/toolbox/database:cast-toolbox-database-models}}\label{\detokenize{contents/toolbox/database::doc}}
\sphinxAtStartPar
The following steps must be taken.

\sphinxAtStartPar
\sphinxstylestrong{Kanishk}, \sphinxstylestrong{Iram} with help from \sphinxstylestrong{Prabhas} to do this.

\sphinxAtStartPar
OK, this is how we do:
\begin{enumerate}
\sphinxsetlistlabels{\arabic}{enumi}{enumii}{}{.}%
\item {} 
\sphinxAtStartPar
Describe data a bit

\item {} 
\sphinxAtStartPar
Provide how to use the code with screenshots

\item {} 
\sphinxAtStartPar
All functions should be explained

\end{enumerate}

\sphinxAtStartPar
We do this for all models.


\chapter{CAST Toolbox \sphinxhyphen{} Analytical Models}
\label{\detokenize{contents/toolbox/an_model/an_model:cast-toolbox-analytical-models}}\label{\detokenize{contents/toolbox/an_model/an_model::doc}}
\sphinxAtStartPar
The following steps must be taken.

\sphinxAtStartPar
\sphinxstylestrong{Sandhya}, \sphinxstylestrong{Iram}, \sphinxstylestrong{Prabhas} and \sphinxstylestrong{Anton} to do this.

\sphinxAtStartPar
\sphinxstylestrong{Sandhya} and \sphinxstylestrong{Iram} \sphinxhyphen{} 2D models
\sphinxstylestrong{Prabhas} Liedl et al 3D
\sphinxstylestrong{Anton} Bioscreen\sphinxhyphen{}AT

\sphinxAtStartPar
OK, this is how we do:
\begin{enumerate}
\sphinxsetlistlabels{\arabic}{enumi}{enumii}{}{.}%
\item {} 
\sphinxAtStartPar
Describe each model \sphinxhyphen{} this we already have

\item {} 
\sphinxAtStartPar
Provide how to use the code with screenshots

\item {} 
\sphinxAtStartPar
Step 2 should talk a bit about input value and about functionalities \sphinxhyphen{} e.g., slider and how to interpret results

\item {} 
\sphinxAtStartPar
We have to do this for both single and multiple scenario mode.

\end{enumerate}

\sphinxAtStartPar
We do this for all models.


\section{CAST Toolbox \sphinxhyphen{} Analytical Models \sphinxhyphen{} Liedl et al. (2005)}
\label{\detokenize{contents/toolbox/an_model/liedl2005:cast-toolbox-analytical-models-liedl-et-al-2005}}\label{\detokenize{contents/toolbox/an_model/liedl2005::doc}}
\sphinxAtStartPar
\sphinxstylestrong{Sandhya} and \sphinxstylestrong{Iram} \sphinxhyphen{} 2D models

\sphinxAtStartPar
OK, this is how we do:
\begin{enumerate}
\sphinxsetlistlabels{\arabic}{enumi}{enumii}{}{.}%
\item {} 
\sphinxAtStartPar
Describe each model \sphinxhyphen{} this we already have

\item {} 
\sphinxAtStartPar
Provide how to use the code with screenshots

\item {} 
\sphinxAtStartPar
Step 2 should talk a bit about input value and about functionalities \sphinxhyphen{} e.g., slider and how to interpret results

\item {} 
\sphinxAtStartPar
We have to do this for both single and multiple scenario mode.

\end{enumerate}

\sphinxAtStartPar
We do this for all models.


\section{CAST Toolbox \sphinxhyphen{} Analytical Models \sphinxhyphen{} Chu et al. (2005)}
\label{\detokenize{contents/toolbox/an_model/chu2005:cast-toolbox-analytical-models-chu-et-al-2005}}\label{\detokenize{contents/toolbox/an_model/chu2005::doc}}
\sphinxAtStartPar
\sphinxstylestrong{Sandhya}, \sphinxstylestrong{Iram}, \sphinxstylestrong{Prabhas} and \sphinxstylestrong{Anton} to do this.

\sphinxAtStartPar
\sphinxstylestrong{Sandhya} and \sphinxstylestrong{Iram} \sphinxhyphen{} 2D models

\sphinxAtStartPar
OK, this is how we do:
\begin{enumerate}
\sphinxsetlistlabels{\arabic}{enumi}{enumii}{}{.}%
\item {} 
\sphinxAtStartPar
Describe each model \sphinxhyphen{} this we already have

\item {} 
\sphinxAtStartPar
Provide how to use the code with screenshots

\item {} 
\sphinxAtStartPar
Step 2 should talk a bit about input value and about functionalities \sphinxhyphen{} e.g., slider and how to interpret results

\item {} 
\sphinxAtStartPar
We have to do this for both single and multiple scenario mode.

\end{enumerate}

\sphinxAtStartPar
We do this for all models.


\section{CAST Toolbox \sphinxhyphen{} Analytical Models \sphinxhyphen{} Ham et al. 2004}
\label{\detokenize{contents/toolbox/an_model/ham2004:cast-toolbox-analytical-models-ham-et-al-2004}}\label{\detokenize{contents/toolbox/an_model/ham2004::doc}}
\sphinxAtStartPar
\sphinxstylestrong{Sandhya}, \sphinxstylestrong{Iram}, \sphinxstylestrong{Prabhas} and \sphinxstylestrong{Anton} to do this.

\sphinxAtStartPar
OK, this is how we do:
\begin{enumerate}
\sphinxsetlistlabels{\arabic}{enumi}{enumii}{}{.}%
\item {} 
\sphinxAtStartPar
Describe each model \sphinxhyphen{} this we already have

\item {} 
\sphinxAtStartPar
Provide how to use the code with screenshots

\item {} 
\sphinxAtStartPar
Step 2 should talk a bit about input value and about functionalities \sphinxhyphen{} e.g., slider and how to interpret results

\item {} 
\sphinxAtStartPar
We have to do this for both single and multiple scenario mode.

\end{enumerate}

\sphinxAtStartPar
We do this for all models.


\section{CAST Toolbox \sphinxhyphen{} Analytical Models \sphinxhyphen{} Liedl et al. 2011}
\label{\detokenize{contents/toolbox/an_model/liedl2011:cast-toolbox-analytical-models-liedl-et-al-2011}}\label{\detokenize{contents/toolbox/an_model/liedl2011::doc}}
\sphinxAtStartPar
\sphinxstylestrong{Prabhas} to do this.

\sphinxAtStartPar
OK, this is how we do:
\begin{enumerate}
\sphinxsetlistlabels{\arabic}{enumi}{enumii}{}{.}%
\item {} 
\sphinxAtStartPar
Describe each model \sphinxhyphen{} this we already have

\item {} 
\sphinxAtStartPar
Provide how to use the code with screenshots

\item {} 
\sphinxAtStartPar
Step 2 should talk a bit about input value and about functionalities \sphinxhyphen{} e.g., slider and how to interpret results

\item {} 
\sphinxAtStartPar
We have to do this for both single and multiple scenario mode.

\end{enumerate}

\sphinxAtStartPar
We do this for all models.


\section{CAST Toolbox \sphinxhyphen{} Analytical Models \sphinxhyphen{} BIOSCREEN\sphinxhyphen{}AT}
\label{\detokenize{contents/toolbox/an_model/bioscreen:cast-toolbox-analytical-models-bioscreen-at}}\label{\detokenize{contents/toolbox/an_model/bioscreen::doc}}
\sphinxAtStartPar
\sphinxstylestrong{Anton} Bioscreen\sphinxhyphen{}AT

\sphinxAtStartPar
OK, this is how we do:
\begin{enumerate}
\sphinxsetlistlabels{\arabic}{enumi}{enumii}{}{.}%
\item {} 
\sphinxAtStartPar
Describe each model \sphinxhyphen{} this we already have

\item {} 
\sphinxAtStartPar
Provide how to use the code with screenshots

\item {} 
\sphinxAtStartPar
Step 2 should talk a bit about input value and about functionalities \sphinxhyphen{} e.g., slider and how to interpret results

\item {} 
\sphinxAtStartPar
We have to do this for both single and multiple scenario mode.

\end{enumerate}

\sphinxAtStartPar
We do this for all models.


\chapter{CAST Toolbox \sphinxhyphen{} Empirical Models}
\label{\detokenize{contents/toolbox/em_model/em_model:cast-toolbox-empirical-models}}\label{\detokenize{contents/toolbox/em_model/em_model::doc}}
\sphinxAtStartPar
The following steps must be taken.

\sphinxAtStartPar
\sphinxstylestrong{Sandhya} and \sphinxstylestrong{Iram}, to do this.

\sphinxAtStartPar
\sphinxstylestrong{Sandhya} and \sphinxstylestrong{Iram} \sphinxhyphen{} 2D models

\sphinxAtStartPar
OK, this is how we do:
\begin{enumerate}
\sphinxsetlistlabels{\arabic}{enumi}{enumii}{}{.}%
\item {} 
\sphinxAtStartPar
Describe each model \sphinxhyphen{} this we already have

\item {} 
\sphinxAtStartPar
Provide how to use the code with screenshots

\item {} 
\sphinxAtStartPar
Step 2 should talk a bit about input value and about functionalities \sphinxhyphen{} e.g., slider and how to interpret results

\item {} 
\sphinxAtStartPar
We have to do this for both single and multiple scenario mode.

\end{enumerate}

\sphinxAtStartPar
We do this for all models.


\section{CAST Toolbox \sphinxhyphen{} Empirical Models \sphinxhyphen{} Birla et al. (2020)}
\label{\detokenize{contents/toolbox/em_model/birla2020:cast-toolbox-empirical-models-birla-et-al-2020}}\label{\detokenize{contents/toolbox/em_model/birla2020::doc}}
\sphinxAtStartPar
\sphinxstylestrong{Sandhya} and \sphinxstylestrong{Iram} to do this. to do this.

\sphinxAtStartPar
OK, this is how we do:
\begin{enumerate}
\sphinxsetlistlabels{\arabic}{enumi}{enumii}{}{.}%
\item {} 
\sphinxAtStartPar
Describe each model \sphinxhyphen{} this we already have

\item {} 
\sphinxAtStartPar
Provide how to use the code with screenshots

\item {} 
\sphinxAtStartPar
Step 2 should talk a bit about input value and about functionalities \sphinxhyphen{} e.g., slider and how to interpret results

\item {} 
\sphinxAtStartPar
We have to do this for both single and multiple scenario mode.

\end{enumerate}

\sphinxAtStartPar
We do this for all models.


\section{CAST Toolbox \sphinxhyphen{} Emperical Models \sphinxhyphen{} Maier and Grathwohl (2005)}
\label{\detokenize{contents/toolbox/em_model/mg2005:cast-toolbox-emperical-models-maier-and-grathwohl-2005}}\label{\detokenize{contents/toolbox/em_model/mg2005::doc}}
\sphinxAtStartPar
The following steps must be taken.

\sphinxAtStartPar
\sphinxstylestrong{Sandhya} and \sphinxstylestrong{Iram} to do this.

\sphinxAtStartPar
\sphinxstylestrong{Sandhya} and \sphinxstylestrong{Iram} \sphinxhyphen{} 2D models
\sphinxstylestrong{Prabhas} Liedl et al 3D
\sphinxstylestrong{Anton} Bioscreen\sphinxhyphen{}AT

\sphinxAtStartPar
OK, this is how we do:
\begin{enumerate}
\sphinxsetlistlabels{\arabic}{enumi}{enumii}{}{.}%
\item {} 
\sphinxAtStartPar
Describe each model \sphinxhyphen{} this we already have

\item {} 
\sphinxAtStartPar
Provide how to use the code with screenshots

\item {} 
\sphinxAtStartPar
Step 2 should talk a bit about input value and about functionalities \sphinxhyphen{} e.g., slider and how to interpret results

\item {} 
\sphinxAtStartPar
We have to do this for both single and multiple scenario mode.

\end{enumerate}

\sphinxAtStartPar
We do this for all models.


\chapter{CAST Toolbox \sphinxhyphen{} Numerical Models}
\label{\detokenize{contents/toolbox/num_model:cast-toolbox-numerical-models}}\label{\detokenize{contents/toolbox/num_model::doc}}
\sphinxAtStartPar
The following steps must be taken.

\sphinxAtStartPar
\sphinxstylestrong{Anton} and \sphinxstylestrong{Prabhas}, to do this.

\sphinxAtStartPar
OK, this is how we do:
\begin{enumerate}
\sphinxsetlistlabels{\arabic}{enumi}{enumii}{}{.}%
\item {} 
\sphinxAtStartPar
Describe the model \sphinxhyphen{} this we already have

\item {} 
\sphinxAtStartPar
Provide how to use the code with screenshots

\item {} 
\sphinxAtStartPar
Step 2 should talk a bit about input value and about functionalities \sphinxhyphen{} e.g., slider and how to interpret results

\end{enumerate}


\chapter{CAST Toolbox \sphinxhyphen{} Model Selection method}
\label{\detokenize{contents/toolbox/mod_sel:cast-toolbox-model-selection-method}}\label{\detokenize{contents/toolbox/mod_sel::doc}}
\sphinxAtStartPar
The following steps must be taken.

\sphinxAtStartPar
\sphinxstylestrong{Prabhas} and \sphinxstylestrong{Natalia} , to do this.

\sphinxAtStartPar
Basically we talk about decision model here


\chapter{CAST Code Structure}
\label{\detokenize{contents/develop/code_structure:cast-code-structure}}\label{\detokenize{contents/develop/code_structure::doc}}
\sphinxAtStartPar
\sphinxstylestrong{Vedanti} will to do this

\sphinxAtStartPar
This basically talks about “code Structuring”

\sphinxAtStartPar
No very detailed info to be added.
Most of them are already there\sphinxhyphen{} or bring from your project report.


\chapter{CAST Code Libraries}
\label{\detokenize{contents/develop/code_libraries:cast-code-libraries}}\label{\detokenize{contents/develop/code_libraries::doc}}
\sphinxAtStartPar
\sphinxstylestrong{Vedanti} will to do this

\sphinxAtStartPar
This basically talks about “Different language and libraries”

\sphinxAtStartPar
No very detailed info to be added.
Most of them are already there\sphinxhyphen{} or bring from your thesis.


\chapter{CAST Code Development}
\label{\detokenize{contents/develop/code_develop:cast-code-development}}\label{\detokenize{contents/develop/code_develop::doc}}
\sphinxAtStartPar
\sphinxstylestrong{Vedanti} will to do this

\sphinxAtStartPar
This basically talks about “code level development”

\sphinxAtStartPar
No very detailed info to be added.
Most of them are already there\sphinxhyphen{} or bring from your thesis.


\chapter{Cite CAST}
\label{\detokenize{contents/ref/cite:cite-cast}}\label{\detokenize{contents/ref/cite::doc}}
\sphinxAtStartPar
The following steps must be taken.

\sphinxAtStartPar
\sphinxstylestrong{Prabhas} will to do this


\chapter{CAST Versions}
\label{\detokenize{contents/ref/version:cast-versions}}\label{\detokenize{contents/ref/version::doc}}
\sphinxAtStartPar
\sphinxstylestrong{Vedanti} with help from \sphinxstylestrong{Prabhas} to do this

\sphinxAtStartPar
Very short one.
We have the first version \sphinxhyphen{} offline/online development
a







\renewcommand{\indexname}{Index}
\printindex
\end{document}